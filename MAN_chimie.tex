\documentclass[10pt,a4paper]{book}
\usepackage[utf8]{inputenc}
\usepackage{amsmath}
\usepackage{amsfonts}
\usepackage{amssymb}
\usepackage{wasysym}
\usepackage{multicol}
\usepackage{hyperref}
\usepackage{tabularx}
\usepackage{subfig}
\usepackage{tikz}
\usepackage[ruled, lined, longend]{algorithm2e}
\usepackage[shortlabels]{enumitem}
\usepackage{textcomp}

\setlength{\parindent}{20pt}
\hypersetup{
    colorlinks,
    citecolor=black,
    filecolor=black,
    linkcolor=black,
    urlcolor=darkgray
}

\newcommand{\R}{\mathbb{R}}
\newcommand{\N}{\mathbb{N}}
\newcommand{\Z}{\mathbb{Z}}
\newcommand{\ind}{\hspace*{\parindent}}

\title{Chimie générale \vspace{0.2cm} - Notes and Summary}
\author{Mahel Coquaz}
\date{Spring Semester 2025}

\begin{document}
\maketitle
\tableofcontents
\newpage
\part*{Atomistique}
\chapter{Atomistique}
\section{L'atome}
\paragraph{Le modèle de l'atome}
Les molécules sont constituées d'atomes qui se partagent des électrons, ces liaisons chimiques dépendent des électrons externes et donc de la configuration électronique des atomes.
Le modèle de l'atome:
\begin{enumerate}
\item Modèle Rutherford
\begin{itemize}
\item L'électron tourne autour du noyau de manière aléatoire.
\item Rendu obsolète.
\end{itemize}
\item Modèle Schrödinger
\begin{itemize}
\item On ne sait pas précisément où est l’électron, c'est un modèle mathématique.
\item Modèle actuel, quantique.
\end{itemize}
\item Modèle Bohr
\begin{itemize}
\item L'électron tourne autour du noyau selon des orbites précises correspondant à des niveaux énergétiques.
\item Physiquement faux mais toujours utilisé pour décrire certaines propriétés atomiques.
\end{itemize}
\item Modèle Thomson
\begin{itemize}
\item Charge positive distribuée uniformément sur une sphère. Les électrons sont distribués de manière à contrebalancer cette charge.
\item Obsolète.
\end{itemize}
\end{enumerate}
\newpage
\paragraph{L'atome et ses constituants} L'atome est constitué de:
\begin{itemize}
\item Le noyau de diamètre \textasciitilde 1 femtomètre (10$^{-15}$m)
\begin{itemize}
\item Protons
\begin{itemize}
	\item Masse: 1.0073 uma\footnote{1 uma = 1.66054 $\times$ 10$^{-24}$ g}
	\item Charge: Positive (+1) 
\end{itemize}
\item Neutrons
\begin{itemize}
	\item Masse: 1.0087 uma
	\item Charge: Neutre (+0) 
\end{itemize}
\end{itemize}
\item Le nuage électronique de diamètre 1 {\AA}ngstrom (1 {\AA} = 10$^{-10}$m)
\begin{itemize}
\item Électrons
\begin{itemize}
	\item Masse: 5.486 $\times$ 10$^{-4}$ uma
	\item Charge: Négative (-1) 
\end{itemize}
\end{itemize}
\end{itemize}
\paragraph{Les atomes d'un élément} Les protons, neutrons, électrons sont les mêmes pour chaque élément. \\ 
Un élément est caractérisé par son nombre de protons (numéro
atomique) \\
Un atome électriquement neutre comporte le même nombre d’électrons
que de protons.\\
Un atome contenant un nombre différent d’électrons et
de protons est appelé ion (monoatomique). On distingue les cations (chargés positivement) des anions (chargés négativement). \\ 
Les isotopes d’un même élément diffèrent par leur nombre de
neutrons. Les isotopes d’un élément ont la même réactivité chimique.\\
La notation d'un atome est la suivante: \\
\textbf{\[{X_A^Z} \hspace{1cm} { }_{Z numero atomique}^{A nombre de masse}\]} %need to fix later
\section{Structure de l'atome}
\subsection{La conception (semi)quantique}
\paragraph{Les travaux de Niels Bohr} Chez Bohr l'énergie d'un électron est quantifiée: ce sont les niveaux d'énergie. \\
Les valeurs permises des niveaux d'énergie sont définies par:
\begin{displaymath}
E_n = - \frac{R_h}{n^2}
\end{displaymath}
$R_h$ = 2.179$\times$10$^{-18}$ J = 13.6eV\footnote{1eV = 1.602$\times$10$^{-19}$C $\times$ 1V}\\
n $\in$ $\mathbb{N}$ \par
Chaque valeur possible pour l'énergie correspond à une trajectoire et une distance noyau-électron. Le niveau n = 1 correspond au niveau d'énergie le plus bas et à l'orbite la plus proche du noyau, c'est \textbf{l'état fondamental}.\par
Les changements d'énergie de l'électron s'opèrent par sauts discontinus et le passe dans un \textbf{état excité}. Tant qu’un électron demeure à un niveau d’énergie donné, il ne peut pas émettre d’énergie sous forme de rayonnement électromagnétique. \par
Lorsque $lim_{n \rightarrow \infty}$ $E_n$ = 0, c'est \textbf{l'ionisation}. \par
\begin{displaymath}
{\Delta}E_n = E_{n_{arriv\acute{e}}} - E_{n_{d\acute{e}part}}
\end{displaymath}
\subsubsection{Résumé du modèle de Bohr}
\begin{enumerate}
\item On a un atome stable.
\item L'énergie d'un électron est quantifié (et quantifiable).
\item Bonne (mais imparfaite) explication du spectre de l’atome d’hydrogène et des atomes avec un seul électron.
\begin{displaymath}
E_n = -\frac{{Z^2}{R_h}}{n^2}
\end{displaymath}
\end{enumerate}
Limitations:
\begin{enumerate}
\item N'explique pas la structure fine des spectres d'hydrogène (manque une information supplémentaire: le spin)
\item Ne s'applique pas aux atomes avec plusieurs électrons (car les intéractions entre électrons sont décrites par la valeur efficace de Z)
\end{enumerate}
\begin{displaymath}
E_n = -\frac{{Z_{eff}^2}{R_h}}{n^2}
\end{displaymath}
\begin{center}
\begin{tabular}{ | m{5cm} | m{5cm}| m{5cm} | } 
  \hline
  Bohr & Schrödinger \\ 
  \hline
  L’électron est décrit comme une particule avec une trajectoire précise. & L'électron est décrit par une fonction d’onde $\psi$ liée à la probabilité de présence. \\ 
  \hline
  cell7 & cell8 \\ 
  \hline
\end{tabular}
\end{center}
\end{document}
