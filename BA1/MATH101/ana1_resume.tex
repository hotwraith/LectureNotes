\documentclass[10pt,a4paper]{book}
\usepackage[utf8]{inputenc}
\usepackage{amsmath}
\usepackage{amsfonts}
\usepackage{amssymb}
\usepackage{wasysym}
\usepackage{multicol}
\usepackage{hyperref}
\usepackage{tabularx}
%\usepackage{subfig}
\usepackage{tikz}
\usepackage[ruled, lined, longend]{algorithm2e}
\usepackage[shortlabels]{enumitem}
\usepackage{textcomp}
\usepackage{chemfig}
\usepackage{mathtools}
\usepackage{dirtytalk}
\usepackage{fancyhdr}
\usepackage{tocloft}

\setlength{\parindent}{20pt}
\hypersetup{
    colorlinks,
    citecolor=black,
    filecolor=black,
    linkcolor=black,
    urlcolor=darkgray
}

\newlistof[chapter]{lectures}{lect}{Cours}

\newcommand{\R}{\mathbb{R}}
\newcommand{\N}{\mathbb{N}}
\newcommand{\Z}{\mathbb{Z}}
\newcommand{\C}{\mathbb{C}}
\newcommand{\x}{$\times$ }
\newcommand{\spa}{\vspace{0.2cm}}
\newcommand{\ind}{\hspace*{\parindent}}
\newcommand{\lecture}[3]{\subsection*{Cours #1 - #2: #3} \label{lecture:#1}
\addcontentsline{lect}{lectures}{Cours #1 - #2}
\fancyhead[R]{\hyperref[lecture:#1]{#3}}
\fancyhead[L]{\hyperref[lecture:#1]{Cours #1 - #2}}
}
%\renewcommand{\lectures}{} % No section numbers 
\newcommand{\lecturetitle}[1]{\nameref{lecture:#1} p.\pageref{lecture:#1}}

\DeclarePairedDelimiter\abs{\lvert}{\rvert}
\DeclarePairedDelimiter\norm{\lVert}{\rVert}

\newtheorem{theorem}{Théorème}[section]
\newtheorem{definition}{Définition}[section]
\newtheorem{proposition}{Proposition}[section]
\newtheorem{corollaire}{Corollaire}[theorem]

\pagestyle{fancy}

\title{Analyse I \vspace{0.2cm} - Résumé}
\author{Mahel Coquaz}
\date{Semestre d'automne 2025}

\begin{document}
\maketitle
\tableofcontents
\newpage
\listoflectures
\newpage

\section*{Introduction}

Ce qui suit se veut être un résumé ultra condensé du cours d'Analyse I pour IN (MATH-101e) donné au semestre d'automne 2025 à l'EPFL. Le contenu de ce cours ne m'appartient pas et est quasiment intégralement extrait du cours des Professeurs Anna Lachowska qui l'a enseigné. J'ai cependant pris la liberté de sauter/raccourcir certains passages et d'ajouter des notes lorsqu'il me semblait pertinent de le faire. \par
Ce résumé/polycopié n'est pas exempt d'erreurs, si vous en trouvez une, vous pouvez me contacter sur mon adresse EPFL \texttt{\href{mailto:mahel.coquaz@epfl.ch}{mahel.coquaz@epfl.ch}} ou via le repo GitHub \url{https://github.com/hotwraith/LectureNotes}. \par
Le repository GitHub est aussi où se trouvent les dernières versions des fichiers PDFs et \TeX pour ce cours (et éventuellement d'autres).

\newpage

\section*{Organisation par cours}

\begin{itemize}
\item \lecturetitle{1}
\item \lecturetitle{2}
\item \lecturetitle{3}
\item \lecturetitle{4}
\end{itemize}

\chapter{Prérequis} 
\lecture{1}{8 septembre 2025}{"C'est trivial ça"}

\section{Identités algébriques}

\begin{itemize}
\item $(x+y)^2 = x^2 + 2xy + y^2$
\item $(x+y)(x-y) = x^2-y^2$
\item $(x-y)(x^2+xy+y^2) = x^3-y^3$
\item $(x+y)(x^2-xy+y^2) = x^3+y^3$
\end{itemize}

\section{Exponentielles \& Logarithmes}

\subsection{Exponentielles}

Avec $a,b \in \R$
\begin{itemize}
\item $a^xa^y = a^{x+y}$
\item $\frac{a^x}{a^y} = a^{x-y}$
\item $(ab)^x = a^xb^x$
\item $a^0 = 1$
\item $(a^x)^y = a^{xy}$
\item $\sqrt[n]{a} = a^{1/n}$
\item $\left(\frac{a}{b}\right)^x = \frac{a^x}{b^x}$
\item $a^1 = a$
\end{itemize}

\subsection{Logarithmes}

Avec $\mathbb{\ln = \log_e}$ le logarithme naturel
\begin{itemize}
\item $\ln(xy) = \ln(x) + \ln(y)$
\item $\ln(\frac{x}{y}) = \ln(x) - \ln(y)$
\item $\ln(x^c) = c\cdot \ln(x)$
\item $\ln(1) = 0$
\item $\log_a(a) = 1$
\end{itemize}

\section{Trigonométrie}

Avec $\sin(x), \cos(x)$  $\forall x \in \R$
\begin{itemize}
\item $\tan x = \frac{\sin x}{\cos x}$ \& $\cot x = \frac{\cos x}{\sin x}$
\item $\sin(x\pm y) = \sin(x)\cos(y) \pm \cos(x)\sin(y)$
\item $\cos(x \pm y) = \cos(x)\cos(y) \mp \sin(x)\sin(y)$
\item $\cos(0) = \cos(x-x) = \cos^2(x) + \sin^2(x) = 1$
\item $\sin(2x) = \sin(x+x) = \sin(x)\cos(x) + \cos(x)\sin(x) = 2\sin(x)\cos(x)$
\item $\cos(2x) = \cos(x+x) = \cos^2(x) - \sin^2(x)$
\end{itemize}

\section{Fonctions élémentaires}

\subsection{Types de fonctions}
\begin{enumerate}
\item Polynomiales
\begin{itemize}
\item Linéaire: $f(x) = ax + b$; $a,b \in \R$
\item Quadratiques: $f(x) = ax^2 + bx + c$; $a,b,c \in \R, a \neq 0$
\end{itemize}
\item Fonctions rationnelles: $f(x) = \frac{P(x)}{Q(x)}$ où $P(x)$ et $Q(x)$ sont des polynômes, et $Q(x) \neq 0$
\item Fonctions algébriques: Toute fonction qui est une solution d'une équation polynomiale, ex: $f(x) = \sqrt{x}$
\item Fonctions transcendantes: fonctions non algébriques
\begin{enumerate}
\item Exponentielles et logarithmiques: $f(x) = e^x$, $g(x) = \ln(x)$
\item Fonctions trigos et réciproques: $f(x) = \sin(x)$, $g(x) = \cos(x)$ 
\end{enumerate}
\end{enumerate}

\subsection{Injectivité, surjectivité, bijectivité}

\begin{definition}
$D(f) = \left\lbrace x \in \R: f(x) \right.$ est bien définie $\left. \right\rbrace$ = le \textbf{domaine de définition} de $f$ \\
$f(D) = \left\lbrace y \in R: \exists x \in D(f): f(x) = y \right\rbrace$ = \textbf{l'ensemble image} de $f$
\end{definition}

\begin{definition} \textbf{Surjectivité}\\
$f: E \rightarrow F$ est \textbf{surjective} si $\forall y \in F, \exists$ \textbf{au \underline{moins} un} $x\in E: f(x) = y$
\end{definition}

\begin{definition} \textbf{Injectivité}\\
$f: E \rightarrow F$ est \textbf{injective} si $\forall y \in F, \exists$ \textbf{au \underline{plus} un} $x\in E: f(x) = y$ \\
Autrement dit: Soit $x_1, x_2 \in D_f: f(x_1) = f(x_2) \rightarrow x_1 = x_2$
\end{definition}

\begin{definition} \textbf{Bijectivité}
Si $f: E \rightarrow F$ est injective \textbf{ET} surjective, alors elle est \textbf{bijective}
\end{definition}

\subsection{Fonctions réciproques}

\begin{definition}
N'existent que si $f: E \rightarrow F$ est \textbf{bijective} et est définie par $f^{-1}: F \rightarrow E$ donc $f(x) = y \Leftrightarrow x = f^{-1}(y)$
\end{definition}

\subsection{Fonctions composées}

Soit $f: D_f \rightarrow \R$ et $g: D_g \rightarrow \R$ avec $f(D_f) \subset D_g$ on peut alors définir la fonction composée $g\circ f : D_f \rightarrow$ par $g\circ f(x) = g(f(x))$ \footnote{Il est bon de noter que de manière générale: $g\circ f \neq f\circ g$}

\chapter{Nombre réels}

\lecture{2}{10 septembre 2025}{For $\R$ ?}

\section{Ensembles}

Un ensemble est une \say{Collection des objets définis et distincts} (G. Cantor) 

\begin{definition} $\mathbf{X \subset Y}$
Soit $\forall b \in X \Rightarrow b \in Y$ \\
Sa négation: $\mathbf{X \not\subset Y}$ \\
$\exists a \in X: a \notin Y$
\end{definition}
\begin{definition}
$X=Y \Leftrightarrow Y\subset X$ et $X\subset Y$
\end{definition}
\begin{definition}
$\O$ l'ensemble vide: $\O=\{\}$ \\
$\forall X: \O \subset X$ \\
$\forall X: X \subset X$
\end{definition}

\subsection{Opération ensemblistes}

\begin{itemize}
\item Réunion: $X\cup Y = \left\lbrace a \in \cup: a\in X \; ou\; a \in Y\right\rbrace$
\item Intersection: $X\cap Y = \left\lbrace a \in \cap: a\in X \; et\; a \in Y\right\rbrace$
\item Différence: $X\backslash Y = \left\lbrace a \in \backslash : a\in X \; et\; a\notin Y\right\rbrace$
\end{itemize}

\paragraph{Propriété} $\mathbf{A\backslash (B\backslash C) =
(A\backslash B) \cup (A\cap C)}$

\section{Nombres naturels, rationnels, réels}

\subsection{Borne inférieure et supérieure}

\begin{definition}
Soit $S\subset \R, S\neq \O$. Alors $a\in \R (b\in \R)$ est un minorant/majorant de $S$ si $\forall x \in S$ on a: $a\leq x \; ou \; x\leq b$ \\
Si $S$ possède un minorant/majorant on dit que $S$ est \textbf{minoré/majoré}. \\
Si $S$ est majoré \textbf{et} minoré, alors $S$ est dit \textbf{borné}.
\end{definition}

\lecture{3}{15 septembre 2025}{Élisabeth Born(é)e}

\subsection{Supremum et infimum}

\begin{theorem}
Tout sous-ensemble non-vide majoré/minoré $S\subset \R$ admet un supremum/infimum qui est unique. \\
\paragraph{Unicité} Si inf/sup$S$ existe alors il est le plus grand minorant/majorant de $S$
\end{theorem}

\subsection{Notations d'intervalles}
Soit $a < b,\; a,b \in \R$. \\
Intervalles bornés
\begin{itemize}
\item $\left\lbrace x\in \R : a\leq x\leq b\right\rbrace = \left[ a,b\right]$ intervalle fermé borné
\item $\left\lbrace x\in \R : a < x < b\right\rbrace = \left] a,b\right[$ intervalle ouvert borné
\item $\left\lbrace x\in \R : a\leq x < b\right\rbrace = \left[ a,b\right[$ intervalle borné ni ouvert ni fermé
\item $\left\lbrace x\in \R : a < x\leq b\right\rbrace = \left] a,b\right]$ intervalle borné ni ouvert ni fermé
\end{itemize}

Intervalles non-bornés:
\begin{itemize}
\item $\left\lbrace x\in \R : x\geq a\right\rbrace = \left[ a,+\infty \right[$ fermé
\item $\left\lbrace x\in \R : x > a\right\rbrace = \left] a,+\infty \right[$ ouvert
\item $\left\lbrace x\in \R : x\leq b\right\rbrace = \left] -\infty , b\right]$ fermé
\item $\left\lbrace x\in \R : x < b\right\rbrace = \left] -\infty , b\right[$ ouvert
\end{itemize}

\section{Nombres complexes}

\lecture{4}{17 septembre 2025}{ça se complique...}

On sait que $x^2=-1$ n'a pas de solutions dans $\R$, alors on introduit $i$ tel que $i^2 = -1$ \footnote{Oui, en maths quand un truc marche pas on invente un truc pour que ça marche, si seulement on pouvait faire ça en exam...}

\subsection{Propriétés des nombres complexes}
Prenons les $\C$\footnote{$\C$ dénote l'ensemble des complexes} de la forme $\left\lbrace z = a +ib\right\rbrace$, où $a,b\in \R$
\begin{itemize}
\item $(+) \; (a+ib)+(c+id) = (a+c) + i(b+d)$
\begin{itemize}
\item $\exists = \in C: 0+0i = 0$ tel que $(a+ib) + 0 + 0i = a+ib\; \forall a,b\in \R$
\item $\exists$ l'opposé pour $(a+ib)$: $(-a + i(-b)) + (a+ib) = 0+0i=0$
\end{itemize}
\item $(\cdot ) (a+ib)\cdot (c+id) = ac-bd + i(ad+bc)$
\begin{itemize}
\item $\exists 1 \in \C:\; 1+0i=1:\; (a+ib)\cdot (1+0i) = a+ib$
\item $z\in \C ,\; z\neq 0 \Rightarrow \exists z^{-1}\in \C: z\cdot z^{-1} = z^{-1}\cdot z = 1$
\item Pour $z = a+ib\in \C^* \Rightarrow z^{-1} = \frac{a-ib}{a^2+b^2}$
\item $z_1(z_2 + z_3) = z_1z_2 + z_1z_3$
\item $\C$ n'est pas ordonné: $i > 0 \Rightarrow i^2 = -1 > 0$ et $i < 0 \Rightarrow (-i)^2 = -1 > 0$, on voit qu'on a $-1>0$ ce qui est absurde.
\end{itemize}
\end{itemize}

\subsection{Les 3 formes de nombres \texorpdfstring{$\C$}{complexes}}

\subsubsection{Forme cartésienne}

$\mathbf{z=a+ib}$, $a,b\in \R$ \\
$z = Re(z) + Im(z)i$ (Re et Im respectivement les parties réelles et imaginaires de $z$) \\
$\abs{z}= \sqrt{(Re(z)^2 + (Im(z))^2} = \sqrt{a^2 + b^2} \geq 0$\footnote{$\abs{z} = 0 \Leftrightarrow z = 0$} \\
Trouver $\varphi \; et\; arg(z)$:
\begin{itemize}
\item  $a>0$ : $arg(z) = \arctan(\frac{b}{a})\; \in \; ]-\frac{\pi}{2}, \frac{\pi}{2}[$ à $2k\pi$ près, $k\in \Z$
\item  $a<0$ : $arg(z) = \arctan(\frac{b}{a})+\pi \; \in \; ]\frac{\pi}{2}, \frac{3\pi}{2}[$ à $2k\pi$ près, $k\in \Z$
\item Si $a=0$:
\begin{itemize}
\item $arg(z) = \frac{\pi}{2}$ si $Im(z)=b > 0$
\item $arg(z) = \frac{3\pi}{2}$ si $Im(z)=b < 0$
\end{itemize}
\end{itemize}

\subsubsection{Forme polaire trigonométrique}

$\mathbf{z = \rho (\cos(\varphi) + i\sin(\varphi)} \; \rho \leq 0, \; \varphi \in \R$ \\
$\abs{z} = \rho \leq 0$
$\rho \neq 0 \Rightarrow \sin(\varphi) = \frac{Im(z)}{\rho}, \; \cos(\varphi) = \frac{Re(z)}{\rho}, \; \tan(\varphi) = \frac{Im(z)}{Re(z)} = \frac{a}{b}$ si $a = Re(z)\neq 0$

\subsubsection{Forme polaire exponentielle}

\begin{equation} \tag{Formule d'Euler}
e^{iy} = \cos(y) + i\sin(y) \\
\end{equation}
\begin{equation*}
z = \rho(\cos(\varphi) + i\sin(\varphi)) = \rho e^{i\varphi}
\end{equation*}

\subsubsection{Les trois formes}

\begin{equation*}
z = Re(z) + Im(z)i = \abs{z}(\cos(arg(z)) + i\sin((arg(z)) = \abs{z}\cdot e^{i\cdot arg(z)}
\end{equation*}
où
\begin{equation*} \tag{module de z}
\abs{z} = \sqrt{(Re(z))^2 + (Im(z))^2} 
\end{equation*}

\begin{equation*} \tag{argument de z}
\begin{split}
\abs{z} \neq 0 \Rightarrow arg(z) &= \arctan\left(\frac{Im(z)}{Re(z)}\right), \; Re(z)>0 \\
test &= \arctan\left(\frac{Im(z)}{Re(z)}\right)+\pi,\; Re(z) < 0 \\
&= \frac{\pi}{2},\; Re(z) = 0,\; Im(z) > 0 \\
&= \frac{3\pi}{2},\; Re(z) = 0,\; Im(z) < 0
\end{split}
\end{equation*}



\end{document}
