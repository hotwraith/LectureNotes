\documentclass[10pt,a4paper]{book}
\usepackage[utf8]{inputenc}
\usepackage{amsmath}
\usepackage{amsfonts}
\usepackage{amssymb}
\usepackage{wasysym}
\usepackage{multicol}
\usepackage{hyperref}
\usepackage{tabularx}
%\usepackage{subfig}
\usepackage{tikz}
\usepackage[ruled, lined, longend]{algorithm2e}
\usepackage[shortlabels]{enumitem}
\usepackage{textcomp}
\usepackage{chemfig}
\usepackage{mathtools}
\usepackage{dirtytalk}
\usepackage{fancyhdr}
\usepackage{tocloft}
\usepackage{datetime}

\setlength{\parindent}{20pt}
\hypersetup{
    colorlinks,
    citecolor=black,
    filecolor=black,
    linkcolor=black,
    urlcolor=darkgray
}

\newlistof[chapter]{lectures}{lect}{Cours}

\newcommand{\R}{\mathbb{R}}
\newcommand{\N}{\mathbb{N}}
\newcommand{\Z}{\mathbb{Z}}
\newcommand{\C}{\mathbb{C}}
\newcommand{\x}{$\times$ }
\newcommand{\fEF}{f:E\rightarrow F}
\newcommand{\gGH}{g:G\rightarrow H}
\newcommand{\fpar}[3]{#1:#2 \rightarrow #3}
\newcommand{\limpar}[2]{\lim_{#1 \rightarrow #2}}
\newcommand{\spa}{\vspace{0.2cm}}
\newcommand{\ind}{\hspace*{\parindent}}
\newcommand{\lecture}[3]{\subsection*{\hyperref[resume:#1]{Cours #1 - #2: #3}} \label{lecture:#1}
\addcontentsline{lect}{lectures}{Cours #1 - #2}
\fancyhead[R]{\hyperref[lecture:#1]{#3}}
\fancyhead[L]{\hyperref[lecture:#1]{Cours #1 - #2}}
}
%\renewcommand{\lectures}{} % No section numbers 
\newcommand{\lecturetitle}[1]{\nameref{lecture:#1} p.\pageref{lecture:#1} \label{resume:#1}}

\DeclarePairedDelimiter\abs{\lvert}{\rvert}
\DeclarePairedDelimiter\norm{\lVert}{\rVert}

\newtheorem{theorem}{Théorème}[section]
\newtheorem{definition}{Définition}[section]
\newtheorem{proposition}{Proposition}[section]
\newtheorem{corollaire}{Corollaire}[theorem]

\pagestyle{fancy}

\title{Analyse I \vspace{0.2cm} - Résumé}
\author{Mahel Coquaz}
\date{Semestre d'automne 2025}

\begin{document}
\maketitle
\tableofcontents
\newpage
\listoflectures
\newpage

\fancyfoot[LO]{\textit{Last compiled:}}
\ddmmyyyydate
\fancyfoot[RO]{\textit{\today \vspace*{0.2cm} \currenttime}}

\section*{Introduction}

Ce qui suit se veut être un résumé condensé du cours d'Analyse I pour IN (MATH-101e) donné au semestre d'automne 2025 à l'EPFL. Le contenu de ce cours ne m'appartient pas et est quasiment intégralement extrait du cours de la Professeur Anna Lachowska qui l'a enseigné. J'ai cependant pris la liberté de sauter/raccourcir certains passages et d'ajouter des notes lorsqu'il me semblait pertinent de le faire. \par
Il faut également noter que la nature de résumé de ce qui suit ne permet pas d'appréhender toutes les notions ou subtilités du cours, rien de ce qui est fait à l'EPFL ne peut être considéré comme "trivial" contrairement à ce que l'ont peut régulièrement entendre dans la bouche des profeusseurs, ce document à donc plus vocation à être un aide mémoire ou complément plutôt qu'un support complet de cours.  \par
Ce résumé/polycopié n'est pas exempt d'erreurs, si vous en trouvez une, vous pouvez me contacter sur mon adresse EPFL \texttt{\href{mailto:mahel.coquaz@epfl.ch}{mahel.coquaz@epfl.ch}} ou via le repo GitHub \url{https://github.com/hotwraith/LectureNotes}. \par
Le repository GitHub est aussi où se trouvent les dernières versions des fichiers PDFs et \TeX pour ce cours (et éventuellement d'autres). 
\linebreak
\par 
Rendons à César ce qui appartient à César, merci à \href{https://github.com/JoachimFavre}{Joachim Favre} et \href{https://github.com/FocusedFaust}{Faust} dont les notes et polycopiés dactylographiés m'ont inspiré dans la réalisation de ces résumés. %TODO: rajouter les liens github

\newpage

\section{Organisation par cours}\footnote{À cause d'une skill issue il faut cliquer sur le \textbf{numéro} de page pour être envoyé sur la section correspondante du pdf.}

\begin{itemize}
\item \lecturetitle{1}
\begin{itemize}
\item Présentation et explications du cours et de sa forme
\item Révisions et passage en revue des prérequis
\end{itemize}
\item \lecturetitle{2}
\begin{itemize}
\item Notations ensemblistes
\item Opérations ensemblistes
\item Nombres \& théorème des bornes inférieure et supérieure
\end{itemize}
\item \lecturetitle{3}
\begin{itemize}
\item Supremum et infimum
\item Notations d'intervalles
\end{itemize}
\item \lecturetitle{4}
\begin{itemize}
\item Arguments et modules de nombres complexes
\item Les 3 formes de nombres complexes
\item Formule de Moivre et puissance de nombres complexes
\end{itemize}
\item \lecturetitle{5}
\begin{itemize}
\item Conjugué d'un complexe
\item Racines de complexes
\item Équations polynomiales dans $\C$
\end{itemize}
\item \lecturetitle{6}
\begin{itemize}
\item Définition des suites
\item Le raisonnement par récurrence
\item Limites de suites
\end{itemize}
\item \lecturetitle{7}
\begin{itemize}
\item Limites finies
\item Quotient de suites polynomiales
\item Théorème des deux gendarmes
\end{itemize}
\item \lecturetitle{8}
\begin{itemize}
\item Suites géométriques
\item Critère d'Alembert
\item Limites infinies \& formes indéterminées
\end{itemize}
\item \lecturetitle{9}
\begin{itemize}
\item Convergence de suites monotones
\item Le nombre $e$
\item Suites définies par récurrence
\end{itemize}
\item \lecturetitle{10}
\begin{itemize}
\item Sous-suites et suites de Cauchy
\item Limite supérieure et inférieure
\item Définition et exemples de séries numériques
\end{itemize}
\item \lecturetitle{11}
\begin{itemize}
\item Convergence absolue
\item Critère de Leibniz et de comparaison
\item Critère de Cauchy et d'Alembert
\end{itemize}
\item \lecturetitle{12}
\begin{itemize}
\item Définition
\item et propriétés de fonctions réelles
\end{itemize}
\item \lecturetitle{13}
\begin{itemize}
\item Limite d'une fonction
\item Caractérisation de la limite d'une fonction à partir des suites \& critère de Cauchy pour les fonctions
\item Opérations algébriques sur les limites et théorème des deux gendarmes pour les fonctions
\end{itemize}
\item \lecturetitle{14}
\begin{itemize}
\item Limites de fonctions composées
\item Limites à $\pm \infty$
\item Limite infinies et formes indeterminées
\end{itemize}
\item \lecturetitle{15}
\begin{itemize}
\item Limites à droite et à gauche
\item Exponentielle et logarithme
\item Fonctions continues
\end{itemize}
\item \lecturetitle{16}
\begin{itemize}
\item Pas de cours, examen blanc.
\end{itemize}
\item \lecturetitle{17}
\begin{itemize}
\item Prolongement par continuité
\item Fonctions continues sur un intervalle
\item Théorème de la valeur intermédiaire (TVI)
\end{itemize}
\end{itemize}

\chapter{Prérequis} 
\lecture{1}{8 septembre 2025}{"C'est trivial ça"}

\section{Identités algébriques}

\begin{itemize}
\item $(x+y)^2 = x^2 + 2xy + y^2$
\item $(x+y)(x-y) = x^2-y^2$
\item $(x-y)(x^2+xy+y^2) = x^3-y^3$
\item $(x+y)(x^2-xy+y^2) = x^3+y^3$
\end{itemize}

\section{Exponentielles \& Logarithmes}

\subsection{Exponentielles}

Avec $a,b \in \R$
\begin{itemize}
\item $a^xa^y = a^{x+y}$
\item $\frac{a^x}{a^y} = a^{x-y}$
\item $(ab)^x = a^xb^x$
\item $a^0 = 1$
\item $(a^x)^y = a^{xy}$
\item $\sqrt[n]{a} = a^{1/n}$
\item $\left(\frac{a}{b}\right)^x = \frac{a^x}{b^x}$
\item $a^1 = a$
\end{itemize}

\subsection{Logarithmes}

Avec $\mathbb{\ln = \log_e}$ le logarithme naturel
\begin{itemize}
\item $\ln(xy) = \ln(x) + \ln(y)$
\item $\ln(\frac{x}{y}) = \ln(x) - \ln(y)$
\item $\ln(x^c) = c\cdot \ln(x)$
\item $\ln(1) = 0$
\item $\log_a(a) = 1$
\end{itemize}

\section{Trigonométrie}

Avec $\sin(x), \cos(x)$  $\forall x \in \R$
\begin{itemize}
\item $\tan x = \frac{\sin x}{\cos x}$ \& $\cot x = \frac{\cos x}{\sin x}$
\item $\sin(x\pm y) = \sin(x)\cos(y) \pm \cos(x)\sin(y)$
\item $\cos(x \pm y) = \cos(x)\cos(y) \mp \sin(x)\sin(y)$
\item $\cos(0) = \cos(x-x) = \cos^2(x) + \sin^2(x) = 1$
\item $\sin(2x) = \sin(x+x) = \sin(x)\cos(x) + \cos(x)\sin(x) = 2\sin(x)\cos(x)$
\item $\cos(2x) = \cos(x+x) = \cos^2(x) - \sin^2(x)$
\end{itemize}

\section{Fonctions élémentaires}

\subsection{Types de fonctions}
\begin{enumerate}
\item Polynomiales
\begin{itemize}
\item Linéaire: $f(x) = ax + b$; $a,b \in \R$
\item Quadratiques: $f(x) = ax^2 + bx + c$; $a,b,c \in \R, a \neq 0$
\end{itemize}
\item Fonctions rationnelles: $f(x) = \frac{P(x)}{Q(x)}$ où $P(x)$ et $Q(x)$ sont des polynômes, et $Q(x) \neq 0$
\item Fonctions algébriques: Toute fonction qui est une solution d'une équation polynomiale, ex: $f(x) = \sqrt{x}$
\item Fonctions transcendantes: fonctions non algébriques
\begin{enumerate}
\item Exponentielles et logarithmiques: $f(x) = e^x$, $g(x) = \ln(x)$
\item Fonctions trigos et réciproques: $f(x) = \sin(x)$, $g(x) = \cos(x)$ 
\end{enumerate}
\end{enumerate}

\subsection{Injectivité, surjectivité, bijectivité}

\begin{definition}
$D(f) = \left\lbrace x \in \R: f(x) \right.$ est bien définie $\left. \right\rbrace$ = le \textbf{domaine de définition} de $f$ \\
$f(D) = \left\lbrace y \in R: \exists x \in D(f): f(x) = y \right\rbrace$ = \textbf{l'ensemble image} de $f$
\end{definition}

\begin{definition} \textbf{Surjectivité}\\
$f: E \rightarrow F$ est \textbf{surjective} si $\forall y \in F, \exists$ \textbf{au \underline{moins} un} $x\in E: f(x) = y$
\end{definition}

\begin{definition} \textbf{Injectivité}\\
$f: E \rightarrow F$ est \textbf{injective} si $\forall y \in F, \exists$ \textbf{au \underline{plus} un} $x\in E: f(x) = y$ \\
Autrement dit: Soit $x_1, x_2 \in D_f: f(x_1) = f(x_2) \rightarrow x_1 = x_2$
\end{definition}

\begin{definition} \textbf{Bijectivité}
Si $f: E \rightarrow F$ est injective \textbf{ET} surjective, alors elle est \textbf{bijective}
\end{definition}

\subsection{Fonctions réciproques}

\begin{definition}
N'existent que si $f: E \rightarrow F$ est \textbf{bijective} et est définie par $f^{-1}: F \rightarrow E$ donc $f(x) = y \Leftrightarrow x = f^{-1}(y)$
\end{definition}

\subsection{Fonctions composées}

Soit $f: D_f \rightarrow \R$ et $g: D_g \rightarrow \R$ avec $f(D_f) \subset D_g$ on peut alors définir la fonction composée $g\circ f : D_f \rightarrow$ par $g\circ f(x) = g(f(x))$ \footnote{Il est bon de noter que de manière générale: $g\circ f \neq f\circ g$}

\chapter{Nombre réels}

\lecture{2}{10 septembre 2025}{For $\R$ ?}

\section{Ensembles}

Un ensemble est une \say{Collection des objets définis et distincts} (G. Cantor) 

\begin{definition} $\mathbf{X \subset Y}$
Soit $\forall b \in X \Rightarrow b \in Y$ \\
Sa négation: $\mathbf{X \not\subset Y}$ \\
$\exists a \in X: a \notin Y$
\end{definition}
\begin{definition}
$X=Y \Leftrightarrow Y\subset X$ et $X\subset Y$
\end{definition}
\begin{definition}
$\O$ l'ensemble vide: $\O=\{\}$ \\
$\forall X: \O \subset X$ \\
$\forall X: X \subset X$
\end{definition}

\subsection{Opération ensemblistes}

\begin{itemize}
\item Réunion: $X\cup Y = \left\lbrace a \in \cup: a\in X \; ou\; a \in Y\right\rbrace$
\item Intersection: $X\cap Y = \left\lbrace a \in \cap: a\in X \; et\; a \in Y\right\rbrace$
\item Différence: $X\backslash Y = \left\lbrace a \in \backslash : a\in X \; et\; a\notin Y\right\rbrace$
\end{itemize}

\paragraph{Propriété} $\mathbf{A\backslash (B\backslash C) =
(A\backslash B) \cup (A\cap C)}$

\section{Nombres naturels, rationnels, réels}

\subsection{Borne inférieure et supérieure}

\begin{definition}
Soit $S\subset \R, S\neq \O$. Alors $a\in \R (b\in \R)$ est un minorant/majorant de $S$ si $\forall x \in S$ on a: $a\leq x \; ou \; x\leq b$ \\
Si $S$ possède un minorant/majorant on dit que $S$ est \textbf{minoré/majoré}. \\
Si $S$ est majoré \textbf{et} minoré, alors $S$ est dit \textbf{borné}.
\end{definition}

\lecture{3}{15 septembre 2025}{Élisabeth Born(é)e}

\subsection{Supremum et infimum}

\begin{theorem}
Tout sous-ensemble non-vide majoré/minoré $S\subset \R$ admet un supremum/infimum qui est unique. \\
\paragraph{Unicité} Si inf/sup$S$ existe alors il est le plus grand minorant/majorant de $S$
\end{theorem}

\subsection{Notations d'intervalles}
Soit $a < b,\; a,b \in \R$. \\
Intervalles bornés
\begin{itemize}
\item $\left\lbrace x\in \R : a\leq x\leq b\right\rbrace = \left[ a,b\right]$ intervalle fermé borné
\item $\left\lbrace x\in \R : a < x < b\right\rbrace = \left] a,b\right[$ intervalle ouvert borné
\item $\left\lbrace x\in \R : a\leq x < b\right\rbrace = \left[ a,b\right[$ intervalle borné ni ouvert ni fermé
\item $\left\lbrace x\in \R : a < x\leq b\right\rbrace = \left] a,b\right]$ intervalle borné ni ouvert ni fermé
\end{itemize}

Intervalles non-bornés:
\begin{itemize}
\item $\left\lbrace x\in \R : x\geq a\right\rbrace = \left[ a,+\infty \right[$ fermé
\item $\left\lbrace x\in \R : x > a\right\rbrace = \left] a,+\infty \right[$ ouvert
\item $\left\lbrace x\in \R : x\leq b\right\rbrace = \left] -\infty , b\right]$ fermé
\item $\left\lbrace x\in \R : x < b\right\rbrace = \left] -\infty , b\right[$ ouvert
\end{itemize}

\section{Nombres complexes}

\lecture{4}{17 septembre 2025}{ça se complique...}

On sait que $x^2=-1$ n'a pas de solutions dans $\R$, alors on introduit $i$ tel que $i^2 = -1$ \footnote{Oui, en maths quand un truc marche pas on invente un truc pour que ça marche, si seulement on pouvait faire ça en exam...}

\subsection{Propriétés des nombres complexes}
Prenons les $\C$\footnote{$\C$ dénote l'ensemble des complexes} de la forme $\left\lbrace z = a +ib\right\rbrace$, où $a,b\in \R$
\begin{itemize}
\item $(+) \; (a+ib)+(c+id) = (a+c) + i(b+d)$
\begin{itemize}
\item $\exists = \in C: 0+0i = 0$ tel que $(a+ib) + 0 + 0i = a+ib\; \forall a,b\in \R$
\item $\exists$ l'opposé pour $(a+ib)$: $(-a + i(-b)) + (a+ib) = 0+0i=0$
\end{itemize}
\item $(\cdot ) (a+ib)\cdot (c+id) = ac-bd + i(ad+bc)$
\begin{itemize}
\item $\exists 1 \in \C:\; 1+0i=1:\; (a+ib)\cdot (1+0i) = a+ib$
\item $z\in \C ,\; z\neq 0 \Rightarrow \exists z^{-1}\in \C: z\cdot z^{-1} = z^{-1}\cdot z = 1$
\item Pour $z = a+ib\in \C^* \Rightarrow z^{-1} = \frac{a-ib}{a^2+b^2}$
\item $z_1(z_2 + z_3) = z_1z_2 + z_1z_3$
\item $\C$ n'est pas ordonné: $i > 0 \Rightarrow i^2 = -1 > 0$ et $i < 0 \Rightarrow (-i)^2 = -1 > 0$, on voit qu'on a $-1>0$ ce qui est absurde.
\end{itemize}
\end{itemize}

\subsection{Les 3 formes de nombres \texorpdfstring{$\C$}{complexes}}

\subsubsection{Forme cartésienne}

$\mathbf{z=a+ib}$, $a,b\in \R$ \\
$z = Re(z) + Im(z)i$ (Re et Im respectivement les parties réelles et imaginaires de $z$) \\
$\abs{z}= \sqrt{(Re(z)^2 + (Im(z))^2} = \sqrt{a^2 + b^2} \geq 0$\footnote{$\abs{z} = 0 \Leftrightarrow z = 0$} \\
Trouver $\varphi \; et\; arg(z)$:
\begin{itemize}
\item  $a>0$ : $arg(z) = \arctan(\frac{b}{a})\; \in \; ]-\frac{\pi}{2}, \frac{\pi}{2}[$ à $2k\pi$ près, $k\in \Z$
\item  $a<0$ : $arg(z) = \arctan(\frac{b}{a})+\pi \; \in \; ]\frac{\pi}{2}, \frac{3\pi}{2}[$ à $2k\pi$ près, $k\in \Z$
\item Si $a=0$:
\begin{itemize}
\item $arg(z) = \frac{\pi}{2}$ si $Im(z)=b > 0$
\item $arg(z) = \frac{3\pi}{2}$ si $Im(z)=b < 0$
\end{itemize}
\end{itemize}

\subsubsection{Forme polaire trigonométrique}

$\mathbf{z = \rho (\cos(\varphi) + i\sin(\varphi)} \; \rho \leq 0, \; \varphi \in \R$ \\
$\abs{z} = \rho \leq 0$
$\rho \neq 0 \Rightarrow \sin(\varphi) = \frac{Im(z)}{\rho}, \; \cos(\varphi) = \frac{Re(z)}{\rho}, \; \tan(\varphi) = \frac{Im(z)}{Re(z)} = \frac{a}{b}$ si $a = Re(z)\neq 0$

\subsubsection{Forme polaire exponentielle}

\begin{equation} \tag{Formule d'Euler}
e^{iy} = \cos(y) + i\sin(y) \\
\end{equation}
\begin{equation*}
z = \rho(\cos(\varphi) + i\sin(\varphi)) = \rho e^{i\varphi}
\end{equation*}

\subsubsection{Les trois formes}

\begin{equation*}
z = Re(z) + Im(z)i = \abs{z}(\cos(arg(z)) + i\sin((arg(z)) = \abs{z}\cdot e^{i\cdot arg(z)}
\end{equation*}
où
\begin{equation*} \tag{module de z}
\abs{z} = \sqrt{(Re(z))^2 + (Im(z))^2} 
\end{equation*}

\begin{equation*} \tag{argument de z}
\begin{split}
\abs{z} \neq 0 \Rightarrow arg(z) &= \arctan\left(\frac{Im(z)}{Re(z)}\right), \; Re(z)>0 \\
&= \arctan\left(\frac{Im(z)}{Re(z)}\right)+\pi,\; Re(z) < 0 \\
&= \frac{\pi}{2},\; Re(z) = 0,\; Im(z) > 0 \\
&= \frac{3\pi}{2},\; Re(z) = 0,\; Im(z) < 0
\end{split}
\end{equation*}

\begin{equation*} \tag{Formule d'Euler}
e^{i\pi} = -1
\end{equation*}

$\forall \rho > 0, \varphi \in \R, n\in \N^*$:
\begin{equation*} \tag{Formule de Moivre}
\begin{split}
(\rho (\cos(\varphi) + i\sin(\varphi))^n &= \rho^n(\cos(n\varphi)+i\sin(n\varphi)) \\
(\rho e^{i\varphi})^n &= \rho^n e^{in\varphi}
\end{split}
\end{equation*}

\lecture{5}{24 septembre 2025}{C'est du français ou des maths ?}

\subsection{Conjugué}

\begin{definition}
$z = a +ib \; \in \C$ alors le conjugué de $z$ est $\overline{z} = a-ib$ \\
$\mathbf{z\overline{z} = \abs{z}^2\in \R}$ \\
En forme polaire le conjugué s'écrit:
\begin{equation*}
\begin{split}
z = \rho(\cos(\varphi) + i\sin(\varphi)) \Rightarrow \overline{z} &= \rho(\cos(\varphi) - i\sin(\varphi)) \\
&= \rho(\cos(-\varphi) + i\sin(-\varphi)) \\
&= \rho e^{-i\varphi}
\end{split}
\end{equation*}
\end{definition}

\subsubsection{Propriétés}
$\forall z\in \C$:
\begin{enumerate}
\item $\overline{z \pm w} = \overline{z} \pm \overline{w}$
\item $\overline{z \cdot w} = \overline{z} \cdot \overline{w}$
\item $\overline{\left(\frac{z}{w}\right)} = \frac{\overline{z}}{\overline{w}}$
\item $\abs{\overline{z}} = \abs{z}$
\item $a = Re(z) = \frac{z + \overline{z}}{2}$
\item $b = Im(z) = \frac{z - \overline{z}}{2i}$
\item $\cos(\varphi) = \frac{e^{i\varphi} + e^{-i\varphi}}{2}$
\item $\sin(\varphi) = \frac{e^{i\varphi} - e^{-i\varphi}}{2i}$
\end{enumerate}

\subsection{Racines de \texorpdfstring{$\C$}{complexes}}

\begin{proposition}
$w = s\cdot e^{i\varphi},\; w\in \C^*$ alors $\forall n\in \N^*$ \\
$\left\lbrace z\in \C^*: z^n = w \right\rbrace = \left\lbrace \sqrt[n]{s} e^{i\cdot \frac{\varphi+2k\pi}{n}},\; k=0,1,...,n-1\right\rbrace$
\end{proposition}

\subsection{Équations polynomiales dans \texorpdfstring{$\C$}{C}}

\subsubsection{Quadratiques}

\begin{equation*}
az^2 + bz + c = 0,\; a,b,c\in \C,\; a\neq 0
\end{equation*}
\begin{equation*}
\begin{split}
z &= \frac{-b \pm \sqrt{b^2-4ac}}{2a} \\
b^2-4ac &= 0 \Rightarrow \; z = -\frac{b}{2a} \\
&\neq 0 \Rightarrow \; 2\; solutions 
\end{split}
\end{equation*}

\subsubsection{Théorème fondamental de l'algèbre}

\begin{theorem}
Tout polynôme $P(z)= a_nz^n + a_{n-1}z^{n-1} + ... + a_1z + a_0$, $a_n, a_{n-1}, ..., a_0\in \C$. \\
\begin{equation*}
\begin{split}
P(z) &= a_n(z-z_1)(z-z_2)...(z-z_n) \; o\grave{u} \; z_1,...,z_n\in \C \\
&= a_n(z-w_1)^{m_1}(z-w_2)^{m_2}...(z-w_p)^{m_p}
\end{split}
\end{equation*}
\end{theorem}

\subsubsection{Polynômes à coefficients réels}

\begin{proposition}
Si $z\in \C$ est une racine de $P(z)$ à coefficients réels, alors $\overline{z}$ l'est aussi. \\
Donc $P(z)=P(\overline{z})=0$ et $(x-z)(x-\overline{z})$ divise le polynôme
\end{proposition}

\chapter{Suites de nombres réels}

\lecture{6}{29 septembre 2025}{Classé sans suite}

\section{Définition}
\begin{definition}
On définit une suite de nombre réels comme une application $f:\N \rightarrow \R$ définie pour tout nombre naturel $(\forall n \leq n_0 \in \N)$
\end{definition}

\begin{definition}
Une suite $(a_n)$ est majorée (minorée) s'il existe un nombre $M(m)\in \R$ tel que $a_n \leq M \; \forall n \in \N$ ($a_n \geq m \; \forall n \in \N$). \\
On dit que la suite est \textbf{bornée} si elle est majorée \textbf{et} minorée.
\end{definition}

\begin{definition}
Une suite $(a_n)$ est croissante (strictement croissante) si $\forall n \in \N$ on a $a_{n+1} \geq a_n \; (a_{n+1} > a_n)$. \\
Une suite $(a_n)$ est décroissante (strictement décroissante) si $\forall n \in \N$ on a $a_{n+1} \leq a_n \; (a_{n+1} < a_n)$. \\
Une suite est dite (strictement) \textbf{monotone} si elle est (strictement) croissante ou (strictement) décroissante.
\end{definition}

\section{Raisonnement par récurrence}

Soit $P(n)$ une proposition dépendant d'un entier naturel $n$, telle que:
\begin{enumerate}
\item \textbf{Initialisation}: $P(n_0)$ est vraie, et...
\item \textbf{Hérédité}: $\forall n \geq n_0$, $P(n)$ implique $P(n+1)$, alors $P(n)$ est \textbf{vraie} pour tout $n\geq n_0$.
\end{enumerate}

\textbf{Il est \textit{très} important de bien démontrer les deux étapes de la récurrence, autrement il est facile d'obtenir une preuve qui est fausse.}

\section{Limite des suites}

\begin{definition}
On dit que la suite $(x_n)$ est \textbf{convergente} et admet pour \textbf{limite} le nombre réel $l\in \R$ si pour tout $\epsilon > 0, \; \exists n_0 \in \N: \forall n\geq n_0$ on a $\abs{x_n - l} \leq \epsilon$. \\
Une suite qui n'est \textbf{pas} convergente est dite \textbf{divergente}.
\end{definition}

\lecture{7}{1 octobre 2025}{C'est limite ça}

\begin{proposition}
Si elle existe, la limite $l$ d'une suite $(a_n)$ est \textbf{unique}.
\end{proposition}
\begin{proposition}
Toute suite convergente est bornée. \textbf{Attention}, la réciproque est fausse, ex: $a_n = (-1)^n$ est bornée mais non convergente.
\end{proposition}

\begin{equation*} \tag{Inégalité triangulaire}
\abs{x+y} \geq \abs{x} + \abs{y}
\end{equation*}

\begin{proposition}
Soient $(a_n)$ et $(b_n)$ deux suites convergentes: $\lim_{n\rightarrow \infty}a_n=a$ et $\lim_{n\rightarrow \infty}b_n=b$.
\begin{enumerate}
\item $\lim_{n\rightarrow \infty}(a_n \pm b_n) = a\pm b$
\item $\lim_{n\rightarrow \infty}(a_n \cdot b_n) = a\cdot b$
\item $\lim_{n\rightarrow \infty}(\frac{a_n}{b_n}) = \frac{a}{b}, \; \forall b \neq 0$
\end{enumerate}
\end{proposition}

%\paragraph{•} Add remarques p.6 slides cours 7

\subsection{Quotient de deux suites polynomiales}

Prenons:
\begin{equation*}
\begin{split}
x_n &= a_pn^p +...+ a_1n + a_0 \\
y_n &= b_qn^q +...+ b_1n + b_0 \\
\lim_{n\rightarrow \infty} \left(\frac{x_n}{y_n}\right) &= 0,\; p < q \\
&= \frac{a_p}{b_q},\; p = q \\
&= diverge,\; p > q
\end{split}
\end{equation*}
L'idée est la suivante:
\begin{equation*}
\frac{x_n}{y_n} = \frac{a_pn^p+...+a_1n+a_0}{b_qn^q+...+b_1n+b_0} = \frac{n^p}{n^q}\cdot \frac{(a_p + a_{p-1}\frac{1}{n}+...+a_0\frac{1}{n^p})}{(b_q + b_{q-1}\frac{1}{n}+...+b_0\frac{1}{n^q})}
\end{equation*}
Le terme de droite tendant vers $\frac{a_p}{b_q}$ et le terme de gauche tendant vers différentes possibilités listées plus haut selon $p$ et $q$.

\subsection{Théorème des deux gendarmes}
Soient $(a_n)$, $(b_n)$, $(c_n)$, trois suites telles que:
\begin{enumerate}
\item $\lim_{n\rightarrow \infty}a_n =\lim_{n\rightarrow \infty}c_n = l$
\item $\exists k\in \N : \forall n \leq k \Rightarrow a_n \geq b_n \geq c_n$
\end{enumerate}
Alors $\lim_{n\rightarrow \infty}b_n = l$


\lecture{8}{6 octobre 2025}{Le roi d'Alembert...}

\subsection{Cas des suites géométriques}
Les suites géométriques ont la forme $a_n=a_0\cdot r^n$, $a_0\in \R$ et $a_0 \neq 0$, $r\in \R$.
\begin{equation*}
\begin{split}
\lim_{n\rightarrow \infty} a_0r^n = 0, &\; \abs{r} < 1 \\
\lim_{n\rightarrow \infty} a_0r^n = a_0, &\; r = 1 \\
\lim_{n\rightarrow \infty} a_0r^n = divergente, &\; \abs{r} > 1\; ou\; r = -1
\end{split}
\end{equation*}

\subsection*{Intermède notation}
$\binom{n}{k} = \frac{n!}{k!(n-k)!}$ avec $0\leq k\leq n$

\subsection{Remarques sur les limites}

\begin{enumerate}
\item Si $\lim_{n\rightarrow \infty}x_n = l\in \R$ alors $\lim_{n\rightarrow \infty}\abs{x_n} = \abs{l}$
\item Si $\lim_{n\rightarrow \infty}\abs{x_n} = 0 \Rightarrow \lim_{n\rightarrow \infty}x_n = 0$
\item Si $\lim_{n\rightarrow \infty}\abs{x_n} = l \neq 0$ n'implique pas la convergence de $x_n$ (ex $a_n = (-1)^n$)
\item Si $(a_n)$ est bornée et $\lim_{n\rightarrow \infty}b_n = 0$, alors $\lim_{n\rightarrow \infty}a_n b_n = 0$
\end{enumerate}

\subsection{Critère de D'Alembert}

Soit $(a_n)$ une suite telle que $a_n \neq 0 \; \forall n\in \N$ et $\lim_{n\rightarrow \infty}\abs{\frac{a_{n+1}}{a_n}} = \rho \geq 0$, alors:
\begin{itemize}
\item Si $\rho < 1 \Rightarrow \lim_{n\rightarrow \infty}a_n = 0$
\item Si $\rho > 1 \Rightarrow \lim_{n\rightarrow \infty}a_n \;diverge$
\end{itemize}

\subsection{Limites infinies}

\begin{definition}
On dit que $(a_n)/(b_n)$ tends vers $+\infty/-\infty$ si $\forall A > 0\; \exists n_0 \in \N: \; \forall n \leq n_0,\; a_n \geq A / b_n \leq -A$ \\
Notation: $\lim_{n\rightarrow \infty}a_n = \infty, \; \lim_{n\rightarrow \infty}b_n = -\infty$ \\
Attention: les suites $(a_n)$ et $(b_n)$ sont \textbf{divergentes}.\\
Propriétés:
\begin{enumerate}
\item $\lim_{n\rightarrow \infty}a_n = \lim_{n\rightarrow \infty}b_n = \infty \Rightarrow \lim_{n\rightarrow \infty}(a_n + b_n) = \infty$
\item $\lim_{n\rightarrow \infty}a_n = \pm \infty$ et $(b_n)$ est bornée $\Rightarrow \lim_{n\rightarrow \infty}(a_n \pm b_n) = \pm \infty$
\item $\lim_{n\rightarrow \infty}b_n = \infty / -\infty$ et $a_n\geq/\leq b_n$ $\Rightarrow \lim_{n\rightarrow \infty}a_n = \infty/-\infty$\footnote{C'est la "règle d'un seul gendarme" ou théorème du chien méchant, ou etc...}
\item $(a_n)$ bornée et $\lim_{n\rightarrow \infty}b_n = \pm \infty \Rightarrow = \lim_{n\rightarrow \infty}\frac{a_n}{b_n} = 0$
\item $\lim_{n\rightarrow \infty}\frac{a_{n+1}}{a_n},\; a_n\neq\forall n \Rightarrow$ alors $(a_n)$ diverge\footnote{Extension du critère d'Alembert}.
\end{enumerate}
\end{definition}

\subsection{Formes indéterminées}

\begin{enumerate}
\item $\infty - \infty$
\item $\frac{\infty}{\infty}$
\item $\frac{0}{0}$
\item $0\cdot \infty$
\end{enumerate}

\lecture{9}{8 octobre 2025}{eeeaaaooo}

\subsection{Convergence de suites monotones}

\begin{theorem}
Toute suite croissante/décroissante qui est majorée/minorée converge vers son supremum/infimum. \par
Toute suite croissante/décroissante qui n'est \textbf{pas} majorée/minorée tend vers $+\infty/-\infty$. \par 
On utilise $(a_n)\uparrow = (a_n)$ est croissante, $(b_n)\downarrow = (b_n)$ est décroissante.
\end{theorem}

\section{Le nombre \texorpdfstring{$e$}{e}}

Soit $(x_n): x_0=1,\; x_n = \left(1+\frac{1}{n}\right)^n \; \forall n \geq 1$ \\
$(y_n): y_0=1,\; y_n= \sum^n_{k=0} \; \forall n \geq 1$, ($y_n=1+\frac{1}{1!}+\frac{1}{2!}+...+\frac{1}{n!}$) \\
Alors:
\begin{enumerate}
\item $x_n \leq y_n \; \forall n\in \N$
\item $y_n \leq 3 \; \forall n\in \N$
\item $(y_n)\uparrow \; \forall n\in \N$
\item $(x_n)\uparrow \; \forall n\in \N$
\end{enumerate}	
Donc $\exists \lim_{n\rightarrow \infty}y_n = l \leq 3$ $\Rightarrow$ $\exists \lim_{n\rightarrow \infty}x_n = l' \leq 3$. \par
On peut prouver ceci par récurrence, on se rend compte qu'en vrai $\lim_{n\rightarrow \infty}x_n = e$.

\begin{definition}
$\lim_{n\rightarrow \infty}\left(1 + \frac{1}{n}\right)^n \overset{def}{=} e$
\end{definition}

\section{Suites définies par récurrence}

Soit $x_0=a\in \R$ et $x_{n+1} = g(x)$ où $g:\R \rightarrow \R$ une fonction.

\begin{proposition} Récurrence linéaire \\
Soit $a_0 \in \R$, $a_{n+1} = qa_n + b$, où $q,b \in \R$. \\
Alors
\begin{enumerate}
\item si $\abs{q} < 1 \Rightarrow (a_n)$ converge vers $\lim_{n\rightarrow \infty}a_n = \frac{b}{1-q}$
\item si $\abs{q} \geq 1 \Rightarrow (a_n)$ diverge sauf si $(a_n)$ est une suite constante.
\end{enumerate}
\end{proposition}

\begin{proposition}
Si $x_0 \in \R$, $x_{n+1} = g(x_n)$ et $g: E\rightarrow E \subset \R$ telle que:
\begin{enumerate}
\item $\exists m, M \in \R: m\leq g(x)\leq M \; \forall x\in E$
\item $g$ est croissante: $\forall x_1,x_2 \in E: x_1 \leq x_2 \Rightarrow g(x_1) \leq g(x_2)$
\end{enumerate}
Alors la suite $(x_n)$, $x_{n+1} = g(x_n)$ est bornée et monotone $\Rightarrow$ convergente. \\
\underline{Remarque:} Si (2) est remplacé par $x_1 \leq x_2 \Rightarrow g(x_1) \leq g(x_2)$ (g décroissante) $\Rightarrow$ alors $(x_n)$ n'est pas monotone (mais peut être convergente).
\end{proposition}

% \subsection{Méthodes pour les suites définies par récurrences} TODO: reread and adapt (cours 9)

\lecture{10}{13 octobre 2025}{Netflix}

\section{Sous-suites de Cauchy}

\begin{definition}
Une sous suite d'une suite $(a_n)$ est une suite $k\mapsto a_{n_k}$ où $k \mapsto n_k$ est suite strictement croissante de nombres naturels.
\end{definition}
Ex: 
\begin{equation*}
\begin{split}
a_n = (-1)^n \; \forall n\in \N \Rightarrow & a_{2k} = (-1)^{2k},\; (a_{2k})\subset (a_n) \lim_{k\rightarrow \infty}a_{2k} = 1 \\
& a_{2k+1} = (-1)^{2k+1},\; (a_{2k+1})\subset (a_n) \lim_{k\rightarrow \infty}a_{2k+1} = -1 \\
& (a_n) \; est\; divergente
\end{split}
\end{equation*}

\begin{proposition} Convergence d'une sous-suite \\
Si $\lim_{n\rightarrow \infty}a_n = l \Rightarrow$ toute sous-suite $(a_{n_k})$ converge aussi vers $l$.
\end{proposition}
\begin{theorem} Théorème de Bolzano-Weierstrass \\
Dans toute suite bornée il existe une sous-suite convergente.
\end{theorem}

\subsection{Suites de Cauchy}

\begin{definition}
La suite $(a_n)$ est une suite de Cauchy si $\forall \epsilon > 0$ il existe $n_0\in \N$ tel que $\forall n\geq n_0$ et $\forall m\geq n_0$, $\abs{a_n - a_m} \leq \epsilon$
\end{definition}
\begin{proposition}
Une suite $(a_n)$ est une suite de Cauchy $\Leftrightarrow$ $(a_n)$ est convergente.
\end{proposition}

\section{Limite supérieure et limite inférieure d'une suite bornée}

\begin{definition}
Soit $(x_n)$ une suite bornée: $\exists m,M\in \R$: $m\leq x_n\leq M \; \forall n\in \N$. 
\begin{align*}
On\; d\acute{e}finit\; & la\; suite\; y_n = sup\left\lbrace x_k, k\geq n\right\rbrace \; y_n\downarrow \; y_n \geq x_n \geq m \; \forall n \in \N \\
& la\; suite\; z_n = inf\left\lbrace x_k, k\geq n\right\rbrace \; y_n\uparrow \; z_n \leq x_n \leq M \; \forall n \in \N
\end{align*}
Alors: \\
$\exists \lim y_n \overset{\mathbf{def}}{=} \lim_{n\rightarrow \infty}\sup x_n$, $(y_n)\downarrow$, minorée par $m$ \\
$\exists \lim z_n \overset{\mathbf{def}}{=} \lim_{n\rightarrow \infty}\inf x_n$, $(z_n)\uparrow$, majorée par $M$ \\
$z_n \leq x_n \leq y_n,\; \forall n\in \N$ \\
\paragraph{Remarque:} Si $\liminf x_n = \limsup x_n = l \Rightarrow \lim_{n\rightarrow \infty} y_n = \lim_{n\rightarrow \infty} z_n = l$. On a, par les deux gendarmes, $\exists \lim_{n\rightarrow \infty} x_n = l$
\paragraph{Remarque} Si $\liminf x_n = l_1$, $\limsup x_n = l_2 \Rightarrow \exists$ une sous-suite $\left\lbrace x_{n_k}\right\rbrace$ convergente vers $l_1$ et une sous-suite $\left\lbrace x_{n_j}\right\rbrace$ convergente vers $l_2$ 
\end{definition}

\chapter{Séries numériques}

\section{Définitions et exemples}

\begin{definition} La série de terme général $a_n$ est un couple:
\begin{enumerate}
\item la suite $(a_n)$
\item La suite des sommes partielles $S_n \overset{\mathbf{def}}{=} \sum^n_{k=0} a_k = a_0 + a_1 + a_2 + ... + a_n$
\end{enumerate}
Notation: 
\begin{itemize}
\item $\sum^{\infty}_{k=0}a_k$: Série de terme général $a_k$
\item $S_n = \sum^{n}_{k=0}a_k$: 'n'ième somme partielle
\end{itemize}
\end{definition}

\begin{definition}
Série $\sum^{\infty}_{k=0}a_k$ est convergente $\overset{\mathbf{def}}{\Leftrightarrow}$ la suite $(S_n)$ des sommes partielles est convergente. \\
\begin{align*}
\lim_{n\rightarrow \infty} S_n &= l \Longrightarrow \sum^{\infty}_{k=0}a_k &= l \\
&= \pm \infty &= \pm \infty
\end{align*}
\end{definition}

\paragraph{Exemples}
\begin{itemize}
\item $\sum^{\infty}_{k=0}r^k = \frac{1}{1-r}\; \; \abs{r} < 1$
\item $\sum^{\infty}_{k=1} \frac{1}{k}= \infty$
\item $\sum^{\infty}_{k=1} \frac{1}{k^2}$ converge
\item $\sum^{\infty}_{k=1} \frac{1}{k^p}\; \; \forall p>1$
\end{itemize}

\lecture{11}{15 octobre 2025}{Critères}

\begin{definition}
Une série $\sum^{\infty}_{k=1} a_k$ est dite \textbf{absolument convergente} si la série $\sum^{\infty}_{k=1} \abs{a_k}$ est convergente.
\paragraph{Proposition:} Une série absolument convergente est convergente.
\paragraph{Proposition:} Si la série $\sum^{\infty}_{n=1}a_n$ converge, alors $\lim_{n\rightarrow \infty}a_n = 0$
\paragraph{Remarque:} $\lim_{n\rightarrow \infty}a_n = 0$ n'implique pas la convergence.
\end{definition}

\section{Critères de convergence}

\subsection{Critère de Leibniz}

\begin{proposition}
Soit $\sum^{\infty}_{n=0}a_n$ une série telle que:
\begin{enumerate}
\item $\exists p\in \N$: $\forall n\geq p \Rightarrow \abs{a_{n+1}} \leq \abs{a_n}$
\item $\exists q\in \N$: $\forall n\geq q \Rightarrow \abs{a_{n+1}}\cdot \abs{a_n} \leq 0$
\item $\lim_{n\rightarrow \infty}a_n = 0$
\end{enumerate}
Alors $\sum^{\infty}_{n=0}a_n$ est convergente.
\end{proposition}

\subsection{Critère de comparaison}

\begin{proposition}
Soit $(a_n)$ et $(b_n)$ deux suites telles que $\exists k\in \N$: $0\leq a_n \leq b_n\; \; \forall n\geq k$
Alors:
\begin{itemize}
\item Si $\sum^{\infty}_{n=0} b_n$ converge $\Longrightarrow \sum^{\infty}_{n=0} a_n$ converge.
\item Si $\sum^{\infty}_{n=0} a_n$ diverge $\Longrightarrow \sum^{\infty}_{n=0} b_n$ diverge.
\end{itemize}
\paragraph{Remarque} Si $\sum^{\infty}_{n=0} a_n$ ne possède que des termes positifs/négatifs, et la suite des sommes partielles est \textbf{majorée/minorée}, alors la série $\sum^{\infty}_{n=0} a_n$ est convergente.
\end{proposition}

\subsection{Critère d'Alembert}

Soit $(a_n)$ une suite: $a_n \neq 0\; \forall n\in \N$ et $\lim_{n\rightarrow \infty}\frac{a_{n+1}}{a_n} = \rho \in \R$
Alors si:
\begin{itemize}
\item $p<1 \Longrightarrow \sum^{\infty}_{n=0}a_n$ converge absolument.
\item $p>1 \Longrightarrow \sum^{\infty}_{n=0}a_n$ diverge.
\item $p=1 \Longrightarrow$ pas de conclusion.
\end{itemize}



\subsection{Critère de Cauchy}\footnote{de la racine}

Soit $(a_n)$ une suite et $\exists \lim_{n\rightarrow \infty} \sqrt[n]{\abs{a_n}} = \rho \in \R$. \\
\begin{align*}
Alors\; si & p<1 \Longrightarrow \sum^{\infty}_{n=0}a_n\; converge\; absolument \\
& p>1 \Longrightarrow \sum^{\infty}_{n=0}a_n \;diverge
\end{align*}



\subsection{Remarques}

\begin{enumerate}
\item Si $\limpar{n}{\infty}\abs{\frac{a_{n+1}}{a_n}}=r$ et $\limpar{n}{\infty}\abs{a_n^{\frac{1}{n}}}=l$ alors $r=l$
\item Parfois $\limpar{n}{\infty}\sqrt[n]{\abs{a_n}}$ existe, mais $\limpar{n}{\infty}\abs{\frac{a_{n+1}}{a_n}}$ n'existe pas.
\item Si $\limpar{n}{\infty}\abs{\frac{a_{n+1}}{a_n}}=1$ ou $\limpar{n}{\infty}\sqrt[n]{\abs{a_n}}=1$, alors pas de conclusion sur la convergence de $\sum^{\infty}_{n=0}a_n$.
\end{enumerate}

\chapter{Fonctions réelles}

\lecture{12}{27 octobre 2025}{"Cours le plus facile" -Lachowska}

\section{Définitions et propriétés}

\begin{definition}
Une fonction $f:E\rightarrow F$ où $E,F\in \R$ est une application qui $\forall x\in D(f)=E$ done un élément $y=f(x)\in F$.\\
On note
\begin{itemize}
\item $D(f) = E$ le domaine de définition
\item $f(D) \in F$ l'ensemble image/d'arrivée
\end{itemize}
\end{definition}

\subsection{Propriétés de base}

\begin{enumerate}
\item $f$ est (\textbf{strictement}) croissante sur $D(f)$ si $\forall x_1, x_2 \in D(f), x_1 < x_2 \Rightarrow f(x_1) \leq f(x_2)\; || \; f(x_1) < f(x_2)$. On note $f(x)\uparrow$ "$f(x)$ croissante".
\item Même logique mais opposée pour (\textbf{strictement}) décroissante, notée $f
(x)\downarrow$ "$f(x)$ décroissante".
\item Si $f$ est (\textbf{strictement}) croissante/décroissante sur $D(f)$, alors elle est (\textbf{strictement}) monotone sur $D(f)$.
\item $f$ est \textbf{paire} si $D(f)$ est symmétrique: $x\in D(f) \Rightarrow -x\in D(f)$ et $f(-x)=f(x)\; \forall x\in D(f)$.
\item $f$ est \textbf{impaire} si $D(f)$ est symmétrique: $x\in D(f)\Rightarrow -x\in D(f)$ et $f(-x)=-f(x)\; \forall x\in D(f)$.
\item $f:E\rightarrow F$ est \textbf{périodique} si $\exists P\in \R^*$ tel que $\forall x\in E \Rightarrow x\pm P\in E$ et $f(x\pm P) = f(x)\; \forall x\in E$.
\item $f:E\rightarrow F$ est \textbf{majorée (minorée)} sur $A\in E$ si l'ensemble $f(A)\in \R$ est \textbf{majoré (minoré)}.\\
Si $f(x)$ est majorée \textbf{et} minorée sur $A$, alors elle est dit bornée sur ce même ensemble.
\item 
\begin{enumerate}
\item borne supérieure $\sup_{x\in A} f(x) \overset{def}{=} \sup\left\lbrace f(x),x\in A\right\rbrace$
\item borne inférieure $\inf_{x\in A} f(x) \overset{def}{=} \inf\left\lbrace f(x),x\in A\right\rbrace$
\end{enumerate}
\item Maximum et minimum local d'une fonction:\\
$f:E\rightarrow F, x_0\in E$. Alors $f$ admet un max/min local au point $x_0$ si $\exists \delta >0: \forall x\in D(f)$ et tels que $\abs{x-x_0} \leq \delta$. \\
On a $f(x)\leq f(x_0)$ (max loc), $f(x)\geq f(x_0)$ (min loc).
\item Maximum et minimum global d'une fonction:\\
Même logique, sauf que cela s'applique $\forall x\in E$.
\textbf{Remarque:} Une fonction bornée (majorée \textbf{et} minorée sur $E$) n'atteint pas forcément son min ou max sur cet intervalle.
\item $f:E\rightarrow F$ est \textbf{surjective} si $\forall y\in F,\; exists$ au moins un $x\in E: f(x)=y$
\item $f:E\rightarrow F$ est \textbf{injective} si $\forall y\in F,\; exists$ au plus un $x\in E: f(x)=y$\\
\item Remarque:
\begin{itemize}
\item Si $\fEF$ n'est pas injective $\Rightarrow$ il faut réduire $E$
\item Si $\fEF$ n'est pas surjective $\Rightarrow$ il faut réduire $F$
\end{itemize}
\item Si $\fEF$ est injective \textbf{et} surjective, alors elle est \textbf{bijective}.
\item Si $\fEF$ est bijective, on peut définir la fonction réciproque par la formule:\\
$y=f(x),\; x\in E \Longleftrightarrow x = f^{-1}(y),\; y\in F$
\item Composition des fonctions: soit $\fEF$ et $\gGH$, $E,F,G,H\in \R$
\par Supposons $f(E)\in G \Rightarrow$ On définit la fonction composée: \\
$g\circ f(x) = g(f(x)): E\longrightarrow H$
\par Supposons $g(G)\in E \Rightarrow$ On définit la fonction composée: \\
$f\circ g(x) = f(g(x)): G\longrightarrow F$
\end{enumerate}

\lecture{13}{29 octobre 2025}{I'm reaching my limit}
\section{Limite d'une fonction}

\begin{definition}
Une fonction $\fEF$ est définie \textbf{au voisinage} de $x_0\in \R$ si $\exists \delta >0: \left\lbrace x\in \R : 0<\abs{x-x_0}<\delta \right\rbrace \in E$\\
\textbf{Remarque:} $f$ n'est pas forcément définie en $x_0$ même.
\end{definition}

\begin{definition}
Une fonction $\fEF$ définie au voisinage de $x_0$ \textbf{admet pour limite} $l\in \R$ lorsque $x\rightarrow x_0$ si $\forall \epsilon >0\; \exists \delta >0:\; \forall x\in E:\; 0<\abs{x-x_0}\leq \delta$ on a $\abs{f(x)-l} \leq \epsilon$
\end{definition}

\subsection{Caractérisation de la limite d'une fonction à partir des suites}

Soit $\fEF$ tel que $\limpar{x}{x_0}f(x) = l \Longleftrightarrow \forall$ suite $(a_n)\in \left\lbrace x\in E,\; x\neq x_0\right\rbrace :\; \limpar{n}{\infty}a_n = x_0$, on a $\limpar{n}{\infty}f(a_n) = l$

\begin{corollaire}
Soit $\fEF$ définie au voisinage de $x_0$.\\
Supposons que $\forall (a_n)\in E\backslash \left\lbrace x_0\right\rbrace:\; \limpar{n}{\infty} a_n = x_0$, la suite $(f(a_n))$ converge. Alors $\limpar{x}{x_0}f(x)$ existe.
\end{corollaire}

\begin{proposition}
Si $\limpar{x}{x_0}f(x) = l_1$ et $\limpar{x}{x_0}f(x) = l_2$, alors $l_1 = l_2$.
\end{proposition}

\subsection{Critère de Cauchy pour les fonctions}

$\exists \limpar{x}{x_0} \Longleftrightarrow \forall \epsilon >0,\; \exists \delta > 0:\: \forall x_1,x_2 \in \left\lbrace x\in E: 0<\abs{x-x_0}\leq \delta\right\rbrace$ on a $\abs{f(x_1)-f(x_2)} \leq \epsilon$

\subsection{Opérations algébriques sur  les limites}

Soit $\fpar{f}{E}{\R}$, $\fpar{g}{E}{\R}$ telles que $\limpar{x}{x_0}f(x)=l_1,\in \R$, $\limpar{x}{x_0}g(x)=l_2,\in \R$, alors:
\begin{enumerate}
\item $\limpar{x}{x_0}(\alpha f(x)+\beta g(x)) = \alpha l_1 + \beta l_2$
\item $\limpar{x}{x_0}(f(x)\cdot g(x)) = l_1\cdot l_2$
\item $\limpar{x}{x_0}(\frac{f(x)}{g(x)}) = \frac{l_1}{l_2}$, si $l_2 \neq 0$
\end{enumerate}

\subsection{Théorème des 2 gendarmes pour les fonctions}

Soient $\fpar{f,g,h}{E}{F}$ telles que:
\begin{enumerate}
\item $\limpar{x}{x_0}f(x)= \limpar{x}{x_0}g(x)=l$
\item $\exists \alpha >0:\forall x\in \left\lbrace x\in E: 0<\abs{x-x_0}\leq \alpha \right\rbrace$, on a $f(x)\leq h(x)\leq g(x)$
\item Alors $\limpar{x}{x_0}h(x) = l$
\end{enumerate}

\lecture{14}{3 novembre 2025}{Infinite number of mathematicians walk into a bar}

\subsection{Théorème: Limite de la composée de deux fonctions}

\begin{theorem}
Soit $fEF$, $\limpar{x}{x_0}f(x)=y_0$; $\gGH ,\limpar{y}{y_0}g(y)=l$.
\par Supposons que $f(E)\in G$ et $\exists \alpha >0: 0<\abs{x-x_0}<\alpha \Rightarrow f(x) \neq y_0$. \\
Alors: $\limpar{x}{x_0}(g\circ f)(x) = l$.
\end{theorem}

\section{Limites lorsque x tend vers \texorpdfstring{$\pm \infty$}{l'infini}}

\begin{definition}
$\fEF$ est définie au voisinage de $\pm \infty$, si $\exists \alpha\in \R :\left] \alpha, +\infty \right[ \in E$ (resp $\left] -\infty ,\alpha \right[ \in E$)
\end{definition}

\begin{definition}
Une fonction $\fEF$ définie au voisinage de $\pm \infty$ \textbf{admet pour limite} $l\in \R$ lorsque $x\rightarrow \pm \infty$ si $\forall \epsilon >0\; \exists \alpha >0:$\\
$\forall x \in E: x\geq \alpha \Rightarrow \abs{f(x)-l} \leq \epsilon$\\
resp. $\forall x \in E: x\leq \alpha \Rightarrow \abs{f(x)-l} \leq \epsilon$
\par Notation: $\limpar{x}{+\infty}f(x) = l$ et $\limpar{x}{-\infty}f(x) = l$
\end{definition}

\section{Limites infinies}

\begin{definition}
$\fEF$ définie au voisinage de $x_0 \in \R$ tend vers $\pm \infty$ lorsque $x\rightarrow x_0$ si $\forall A>0,\; \exists \delta >0:\; 0<\abs{x-x_0}\leq \delta \Rightarrow f(x)\geq A (f(x) \leq -A)$
\par Notation: $\limpar{x}{x_0}f(x) = \pm \infty$ et $\limpar{x}{\pm \infty}f(x) = \pm \infty$
\end{definition}

\subsection{Formes indeterminées}

\begin{itemize}
\item $\infty - \infty$
\item  $\frac{\infty}{\infty}$
\item $\frac{0}{0}$
\item $0\cdot \infty$
\item $0^0$
\item $1^{\infty}$
\item $\infty^0$
\end{itemize}


\lecture{15}{5 novembre 2025}{NàJ}

\subsection{Propriétés des limites infinies}

\begin{enumerate}
\item $\limpar{x}{x_0}f(x) = +\infty\; (-\infty)$ et $\limpar{x}{x_0}g(x) = +\infty\; (-\infty) \Rightarrow \limpar{x}{x_0}(f(x) + g(x)) = +\infty\; (-\infty)$
\item Si $\limpar{x}{x_0}f(x) = \pm \infty$ et si $g(x)$ est bornée autour de $x_0$. Alors $\limpar{x}{x_0}(f(x)+g(x)) = \pm \infty$
\item Si $\limpar{x}{x_0}f(x) = +\infty$ et si $\limpar{x}{x_0}g(x) = l \neq 0$. Alors $\limpar{x}{x_0}(f(x)\cdot g(x)) = +\infty\;\; (-\infty)$ si $l>0\;\; (l<0)$ respectivement.
\item Si $\limpar{x}{x_0}f(x) = \pm \infty$ alors $\limpar{x}{x_0}\frac{1}{f(x)} = 0$
\item Si $\limpar{x}{x_0}f(x)=0$ et $f(x)\neq 0$ au voisinage de $x_0$, alors $\limpar{x}{x_0}\frac{1}{f(x)}= +\infty\;\; (-\infty)$ si $f(x)>0 (f(x)<0)$ au voisinage de $x_0$ respectivement.
\item Si $\limpar{x}{x_0}f(x)= +\infty\;\; (-\infty)$ et qu'au voisinage de $x_0$ on a $g(x)\geq f(x)\;\; (g(x)\leq f(x))$ alors, $\forall x$ au voisinage de $x_0$ $\Rightarrow \limpar{x}{x_0}g(x)= +\infty\;\; (-\infty)$
\item Les propriétés (1) à (6) sont également valables pour $x\rightarrow \pm \infty$
\end{enumerate}

\section{Limites à droite et à gauche}

\begin{definition}
$\fEF$ est définie à droite (gauche) de $x_0$ s'il existe $\alpha > 0$ tel que: $\left]x_0, x_0+\alpha\right[ \subset E$ $(\left]x_0-\alpha, x_0\right[ \subset E)$
\end{definition}
On dénote la limite:
\begin{itemize}
\item à droite: $\limpar{x}{x_0^+}f(x) = l$
\item à gauche: $\limpar{x}{x_0^-}f(x) = l$
\end{itemize}

Remarques:
\begin{itemize}
\item $\limpar{x}{x_0}f(x)=l \Leftrightarrow \limpar{x}{x_0^+}f(x)=l$ et $\limpar{x}{x_0^-}f(x)=l$
\item On peut aussi définir les limites $\limpar{x}{x_0^+}f(x)=\pm \infty$ et $\limpar{x}{x_0^-}f(x)=\pm \infty$.\footnote{Par exemple $f(x)=\frac{1}{x}$ dont les limites en $0^+$ et $0^-$ sont respectivement à $+\infty$ et $-\infty$}
\end{itemize}

\section{Fonction exponentielle et logarithmique}

\subsection{Propriétés et définitions de l'exponentielle}
\begin{equation} \tag{Définition de l'exponentielle}
e^x \overset{\mathbf{d\acute{e}f}}{=} \sum^\infty_{n=0}\frac{x^n}{n!}
\end{equation}
Par convention on admet: $0^0=1$ et $0!=1$ %TODO ajouter l'origine des ces trucs (page 7, cours 15 2025)
\begin{proposition}
\begin{enumerate}
\item $e^{x+y} = e^x \cdot e^y,\forall x,y \in \R$
\item $e^{-x} = \frac{1}{e^x}, \forall x \in \R$
\item $e^x > 0, \forall \in \R$
\end{enumerate}
\end{proposition}
\subsubsection{Propriétés de \texorpdfstring{$f(x)=e^x$}{la fonction e}}
\begin{enumerate}
\item $\limpar{x}{+\infty}e^x=+\infty$
\item $\limpar{x}{-\infty}e^x = \limpar{y}{+\infty}e^{-y}=\limpar{y}{+\infty}\frac{1}{e^y}=0$
\item $e^x\uparrow$ est strictement croissante $\forall x\in \R$
\item $\limpar{x}{a}\frac{e^{t(x)}-1}{t(x)}$ si $\limpar{x}{a}t(x)=0$ et $t(x)\neq 0$ au voisinage de $x=a$
\end{enumerate}

\subsection{Propriétés et définitions du logarithme}

\begin{definition}
Comme $\fpar{e^x}{\R}{\R_+^*}$ est bijective, on peut définir la fonction réciproque, $\fpar{\ln(x)}{\R_+^*}{\R}$.
\end{definition}

\subsubsection{Propriétés}
\begin{enumerate}
\item $e^{\ln(x)}=x, \forall x\in \R_+^*$
\item $\ln(e^x)=x, \forall x\in \R$
\item $\ln(x\cdot y)= \ln(x) + \ln(y), \forall x,y\in \R_+^*$
\item $\ln(\frac{x}{y}) = \ln(x) - \ln(y)$
\item $\ln(x^r) = r\cdot \ln(x), \forall r\in \R$
\item $\ln(1) = 0,\; \ln(e)=1,\; e^0=1,\; e'=e$
\end{enumerate}

\section{Fonctions continues}

\begin{definition}
Une fonction $\fEF$ est continue en un point $x_0\in E$ si $\limpar{x}{x_0}f(x)=f(x_0)$, soit:
\begin{enumerate}
\item $\limpar{x}{x_0}f(x)\in \R$ existe
\item $f(x_0)$ existe
\item Les deux valeurs sont égales.
\end{enumerate}
\end{definition}

\subsection{Quelques fonctions continues remarquables}

\begin{enumerate}
\item $f(x)=x^p, p\in \N$ est continue sur $\R$: $\limpar{x}{a}x^p = a^p,\; \forall a\in \R$
\item Tout polynôme est continue sur $\R$
\item Toute fonction rationelle est continue sur son domaine
\item $f(x)=\sqrt[p]{x}$ est continue sur son domaine $\forall p\in \N$ ($\limpar{x}{a}\sqrt{x}=\sqrt{a}, a>0$)
\item $f(x)=\sin(x)$ et $f(x)=\cos(x)$ sont continues sur $\R$. $f(x)=\tan(x)=\frac{\sin(x)}{\cos(x)}$ et $f(x)=\cot(x)=\frac{\cos(x)}{\sin(x)}$ sont continues sur leurs domaines de définition
\end{enumerate}

\subsection{Limites remarquables}

\begin{itemize}
\item $\limpar{x}{0}\frac{e^x-1}{x}=1$
\item $\limpar{x}{0}\frac{\ln(1+x)}{x}=1$
\item $\limpar{x}{0}(1+x)^{\frac{1}{x}}=e$
\item $\limpar{x}{a}\frac{e^{t(x)}-1}{t(x)}=1$ si $\limpar{x}{a}t(x)=0$
\item $\limpar{x}{a}\frac{\ln(1+t(x)}{t(x)}=1$ si $\limpar{x}{a}t(x)=0$
\item $\limpar{x}{a}(1+t(x))^{\frac{1}{t(x)}}=e$ si $\limpar{x}{a}t(x)=0$
\end{itemize}

\lecture{16}{10 novembre 2025}{C'est un peu beaucoup trop calme}

\lecture{17}{12 novembre 2025}{•}

\subsection{Opérations sur les fonctions continues}

Si les fonctions $f$ et $g$ sont continues en $x=x_0$, alors:
\begin{enumerate}
\item $\alpha f + \beta g$ est continue en $x=x_0$, $\forall \alpha,\beta \in \R$
\item $f\cdot g$ est continue en $x=x_0$
\item $\frac{f}{g}$ est continue en $x=x_0$ si $g(x_0)\neq 0$
\item Si $\fEF$ et $\gGH$, $f(E)\subset G$, et $f$ est continue en $x_0\in E$, $g$ est continue en $f(x_0)\in G$. Alors $(g\circ f)$ est continue en $x_0$
\end{enumerate}

\subsection{Prolongement par continuité d'une fonction en un point}

\begin{definition}
Soit $\fEF$ une fonction telle que $c\notin E$ ($f$ n'est pas définie en $x=c$) et $\limpar{x}{c}f(x)\in \R$ existe. Alors la fonction $\hat{f}:E\cup \left\lbrace c \right\rbrace \rightarrow \R$.
\begin{equation*}
\hat{f}(x)\overset{\mathbf{d\acute{e}f}}{=}\left\lbrace \begin{array}{rcl} f(x),& x\in E \\ \limpar{x}{c}f(x),& x=c \end{array} \right.
\end{equation*}
est appelée \textbf{le prolongement par continuité} de $f$ au point $x=c$. Un tel prolongement est unique et la fonction $\hat{f}$ est continue en $x=c$
\end{definition}

\section{Fonctions continues sur un intervalle}

\begin{definition}
Une fonction $\fpar{f}{I}{F}$, où I est un intervalle ouvert non-vide, est continue si $f$ est continue en tout point $x\in I$. \\
$\fpar{f}{\left[ a,b\right]}{F}$ est continue sur $\left[ a,b\right]$ si elle est continue sur $\left] a,b\right[$ et continue à gauche en $x=b$ et à droite $x=a$.
\end{definition}

\begin{theorem}
Soit $a<b\in \R$ et $\fpar{f}{\left[ a,b\right]}{F}$ une \textbf{fonction continue sur l'intervalle fermé et borné} $\left[ a,b\right]$. Alors $f$ atteint son infimum et son supremum sur $\left[ a,b\right]$ ($\Leftrightarrow \exists \max_{[a,b]}f(x)$ et $\exists \min_{[a,b]}f(x)$)
\end{theorem}

\subsection{Théorème de la valeur intermédiaire}

Soit $a<b\in \R$, $\fpar{f}{[a,b]}{\R}$ une fonction \underline{continue}. Alors $f$ atteint son $\sup$, son $\inf$ et \textbf{\underline{toute valeur comprise entre les deux}}.
\begin{equation*}
f(\left[a,b\right])=\left[\min_{\left[a,b\right]}f(x),\; \max_{\left[a,b\right]}f(x) \right]
\end{equation*}

\begin{corollaire}
Soit $a<b\in \R$ et $\fpar{f}{[a,b]}{\R}$ une fonction continue telle que $f(a)\cdot f(b)<0$ (donc $f(a)$ et $f(b)$ de signes différents).\par
Alors il existe au moins un point $c\in ]a,b[:\; f(c)=0$
\end{corollaire}


%“Your faces look just like mine did when I first took Analysis and was introduced to the epsilon. They said “do you understand?”, of course I said yes…. I was LYING.”
%— Analysis 1 lecturer
%https://mathprofessorquotes.com/
\end{document}
